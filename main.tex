%%%%%%%%%%%%%%%%%%%%%%%%%%%%%%%%%%%%%%%
% This is a modified ONE COLUMN version of
% the following template:
% 
% Deedy - One Page Two Column Resume
% LaTeX Template
% Version 1.1 (30/4/2014)
%
% Original author:
% Debarghya Das (http://debarghyadas.com)
%
% Original repository:
% https://github.com/deedydas/Deedy-Resume
%
% IMPORTANT: THIS TEMPLATE NEEDS TO BE COMPILED WITH XeLaTeX
%
% This template uses several fonts not included with Windows/Linux by
% default. If you get compilation errors saying a font is missing, find the line
% on which the font is used and either change it to a font included with your
% operating system or comment the line out to use the default font.
% 
%%%%%%%%%%%%%%%%%%%%%%%%%%%%%%%%%%%%%%
% 
% TODO:
% 1. Integrate biber/bibtex for article citation under publications.
% 2. Figure out a smoother way for the document to flow onto the next page.
% 3. Add styling information for a "Projects/Hacks" section.
% 4. Add location/address information
% 5. Merge OpenFont and MacFonts as a single sty with options.
% 
%%%%%%%%%%%%%%%%%%%%%%%%%%%%%%%%%%%%%%
%
% CHANGELOG:
% v1.1:
% 1. Fixed several compilation bugs with \renewcommand
% 2. Got Open-source fonts (Windows/Linux support)
% 3. Added Last Updated
% 4. Move Title styling into .sty
% 5. Commented .sty file.
%
%%%%%%%%%%%%%%%%%%%%%%%%%%%%%%%%%%%%%%%
%
% Known Issues:
% 1. Overflows onto second page if any column's contents are more than the
% vertical limit
% 2. Hacky space on the first bullet point on the second column.
%
%%%%%%%%%%%%%%%%%%%%%%%%%%%%%%%%%%%%%%

\documentclass[]{deedy-resume-openfont}


\begin{document}

%%%%%%%%%%%%%%%%%%%%%%%%%%%%%%%%%%%%%%
%
%     LAST UPDATED DATE
%
%%%%%%%%%%%%%%%%%%%%%%%%%%%%%%%%%%%%%%
\lastupdated

%%%%%%%%%%%%%%%%%%%%%%%%%%%%%%%%%%%%%%
%
%     TITLE NAME
%
%%%%%%%%%%%%%%%%%%%%%%%%%%%%%%%%%%%%%%


\namesection{Pranesh}{Mukhopadhyay}{ \urlstyle{same}\url{http://example.com} \\
\href{mailto:user@example.com}{user@example.com} | 96747-70912
}

\href{http://linkedin.com/in/12346}{http://linkedin.com/in/praneshmukhopadhyay}

\begin{center}
\huge\color{subheadings}\custombold{LIFELONG LEARNER}
\end{center}

\section{Experience}

\runsubsection{Coursera}
\descript{| KPCB Fellow + Software Engineering Intern }
\location{Expected June 2014 – Sep 2014 | Mountain View, CA}
\vspace{\topsep} % Hacky fix for awkward extra vertical space
\begin{tightemize}
\item 52 out of 2500 applicants chosen to be a KPCB Fellow 2014.
\end{tightemize}
\sectionsep

\runsubsection{Google}
\descript{| Software Engineering Intern }
\location{May 2013 – Aug 2013 | Mountain View, CA}
\begin{tightemize}
\item Worked on the YouTube Captions team in primarily vanilla Javascript and Python to plan, design and develop the full stack implementation of a new framework to add and edit Automatic Speech Recognition captions.
\item Created a backbone.js-like framework for the Captions editor.
\item All code was reviewed, perfected, and pushed to producti`on.
\end{tightemize}
\sectionsep

\runsubsection{Phabricator}
\descript{| Open Source Contributor \& Team Leader}
\location{Jan 2013 – May 2013 | Palo Alto, CA \& Ithaca, NY}
\begin{tightemize}
\item Phabricator is used daily by Facebook, Dropbox, Quora, Asana and more.
\item I created the Meme generator, the entire Lipsum application, ported Tokens to different apps, fixed many bugs and more in PHP and Shell.
\item Led a team from MIT, Cornell, IC London and UHelsinki for the project.
\end{tightemize}
\sectionsep

%%%%%%%%%%%%%%%%%%%%%%%%%%%%%%%%%%%%%%
%     RESEARCH
%%%%%%%%%%%%%%%%%%%%%%%%%%%%%%%%%%%%%%

\section{Research}
\runsubsection{Cornell Robot Learning Lab}
\descript{| Head Undergrad Research}
\location{Jan 2014 – Present | Ithaca, NY}
Worked with \textbf{\href{http://www.cs.cornell.edu/~ashesh/}{Ashesh Jain}} and \textbf{\href{http://www.cs.cornell.edu/~asaxena/}{Prof Ashutosh Saxena}} to create \textbf{PlanIt}, a tool which  learns from large scale user preference feedback to plan robot trajectories in human environments.  Publication submitted.
\sectionsep

\runsubsection{Cornell Phonetics Lab}
\descript{| Head Undergraduate Researcher}
\location{Mar 2012 – May 2013 | Ithaca, NY}
Lead the development of \textbf{QuickTongue}, the first ever breakthrough tongue-controlled game with \textbf{\href{http://conf.ling.cornell.edu/~tilsen/}{Prof Sam Tilsen}} to aid in Linguistics research. Publication submitted.
\sectionsep

\section{Education}
\runsubsection{Cornell University}
\descript{| MEng in Computer Science}
\location{Expected Dec 2014 | Ithaca, NY \textbullet{} Cum. GPA: N/A}
\sectionsep

\runsubsection{Cornell University}
\descript{| BS in Computer Science}
\location{Conc. in Software Engineering,
College of Engineering | Expected May 2014 | Ithaca, NY}
Dean's List (All Semesters) \textbullet{}
Cum. GPA: 3.92 / 4.0 \textbullet{}
Major GPA: 3.94 / 4.0
\sectionsep

\section{Languages}
\begin{minipage}[t]{.6\textwidth}
\subsection{Programming}
\location{Over 5000 lines:}
Java \textbullet{}   Shell \textbullet{} JavaScript \textbullet{} Matlab \textbullet{}
OCaml \textbullet{} Python \textbullet{} Rails \textbullet{} \LaTeX\ \\ 
\location{Over 1000 lines:}
C \textbullet{} C++ \textbullet{} CSS \textbullet{} PHP \textbullet{} Assembly \\
\location{Familiar:}
AS3 \textbullet{} iOS \textbullet{} Android \textbullet{} MySQL
\sectionsep
\end{minipage}
\hfill
\begin{minipage}[t]{.35\textwidth}
\subsection{Spoken \& Written}
\location{Native fluency:} English, Spanish\\
\location{Reading fluency:} Chinese, Japanese\\
\end{minipage}

\end{document}  \documentclass[]{article}